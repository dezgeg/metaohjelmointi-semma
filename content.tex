\section{Johdanto}
\section{Haskell}
\subsection{Yleistä}
Haskell on staattisesti- ja vahvasti tyypitetty laiska, puhtaasti funktionaalinen ohjelmointikieli.

Haskell-kielen suunnittelu ja kehitys alkoi vuonna 1987,
jolloin päämääränä oli tuoda lukuisten, hyvin samankaltaisten laiskojen puhtaasti funktionaalisten tutkimuskielien ideat yhteen.
Tällä hetkellä kielen uusin spesifikaatio ``Haskell Language Report'' on vuodelta 2010.~\cite{HaskellReport2010}.

Aktiivisesti kehitettyjä sekä ajantasaisia Haskell-kääntäjiä on käytännössä ainoastaan
GHC, Glasgow Haskell Compiler~\cite{GHC}.
Itse kielen käyttäjiä varten julkaistaan säännöllisesti Linux-, Windows- ja
Mac OS X-käyttöjärjestelmille  jakelupaketti ``Haskell Platform''~\cite{HaskellPlatform},
joka sisältää GHC:n lisäksi kirjastoja esimerkiksi yksikkötestaukseen, POSIX-rajapintoihin
sekä OpenGL:ään.

\subsection{Syntaksi ja määrittelyt}
\subsection{Sisäänrakennetut tyypit}
\subsection{Kontrollirakenteet}
\subsection{Omien tyyppien määrittely}
\subsection{Tyyppiluokat}

\section{Metaohjelmointi Haskellilla}
\subsection{Template Haskell}

Template Haskell~\cite{ThPaper} on GHC-kääntäjän toteuttama epävirallinen laajennos Haskell-kieleen,
joka tarjoaa mahdollisuudet käännösaikaiseen metaohjelmointiin makrojen muodossa.

Makrojärjestelmät voidaan karkeasti jakaa tekstuaalisiin makroihin sekä syntaktisiin makroihin.
Tunnetuimpia esimerkkejä tekstuaalisista makroista on C-kielen esikääntäjä sekä erinäisten assemblereiden makrokielet,
kun taas syntaktiset makrot ovat Lisp-kieliperheen tavaramerkki.
Template Haskellin tarjoamat makrot lukeutuvat näistä kategorioista jälkimmäiseen:
Template Haskell-makrot ovat tavallisia Haskell-funktioita,
joilla on paluutyyppinä Template Haskell-kirjastossa määritelty syntaksipuutyyppi.
